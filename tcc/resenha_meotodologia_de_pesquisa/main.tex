\documentclass[
	% -- opções da classe memoir --
	12pt,				% tamanho da fonte
	openright,			% capítulos começam em pág ímpar (insere página vazia caso preciso)
	oneside,			% para impressão em verso e anverso. Oposto a oneside
	a4paper,			% tamanho do papel.
	% -- opções da classe abntex2 --
	chapter=TITLE,		% títulos de capítulos convertidos em letras maiúsculas
	%section=TITLE,		% títulos de seções convertidos em letras maiúsculas
	subsection=TITLE,	% títulos de subseções convertidos em letras maiúsculas
	%subsubsection=TITLE,% títulos de subsubseções convertidos em letras maiúsculas
	% -- opções do pacote babel --
	english,			% idioma adicional para hifenização
	brazilian,				% o último idioma é o principal do documento
	]{abntex2}

% ---
% PACOTES
% ---

% ---
% Pacotes fundamentais
% ---
\usepackage{lmodern}			% Usa a fonte Latin Modern
\usepackage[T1]{fontenc}		% Selecao de codigos de fonte.
\usepackage[utf8]{inputenc}		% Codificacao do documento (conversão automática dos acentos)
\usepackage{indentfirst}		% Indenta o primeiro parágrafo de cada seção.
\usepackage{color}				% Controle das cores
\usepackage{graphicx}			% Inclusão de gráficos
\usepackage{microtype} 			% para melhorias de justificação
% ---

%--------------Pacotes de citações
\usepackage[brazilian,hyperpageref]{backref}	 % Paginas com as citações na bibl
\usepackage[alf]{abntex2cite}	% Citações padrão ABNT

\usepackage{hyperref}

\renewcommand{\backrefpagesname}{Citado na(s) página(s):~}
\renewcommand{\backref}{}
\renewcommand*{\backrefalt}[4]{
	\ifcase #1 %
		Nenhuma citação no texto.%
	\or
		Citado na página #2.%
	\else
		Citado #1 vezes nas páginas #2.%
	\fi}%

\definecolor{blue}{RGB}{41,5,195}

\makeatletter
\hypersetup{
     	%pagebackref=true,
		pdftitle={\@title},
		pdfauthor={\@author},
    	pdfsubject={\imprimirpreambulo},
	    pdfcreator={LaTeX with abnTeX2},
		pdfkeywords={abnt}{latex}{abntex}{abntex2}{plano de trabalho},
		colorlinks=true,       		% false: boxed links; true: colored links
    	linkcolor=blue,          	% color of internal links
    	citecolor=blue,        		% color of links to bibliography
    	filecolor=magenta,      		% color of file links
		urlcolor=blue,
		bookmarksdepth=4
}
\makeatother

\setlength{\parindent}{1.3cm}

\setlength{\parskip}{0.2cm}  % tente também \onelineskip
\makeindex

\begin{document}

\frenchspacing

\textbf{Graduação em Ciência da Computação - UFU}

\textbf{Disciplina:} GBC072 - Projeto de Graduação 1

\textbf{Professor:} Prof. Dr. Rodrigo Sanches Miani

\textbf{Nome:} Heitor Freitas Ferreira

% ------------------ Título -----------------------
\section*{\centerline{\Large \textbf{Entrega 2 (E2) - Resenha crítica}}}

%------------------- Corpo de dados --------------
\vspace*{0.5cm}

\begin{center}
	\textbf{Uma Reflexão sobre a Pesquisa em Ciência da
		Computação à Luz da Classificação das
		Ciências e do Método Científico}
\end{center}

\section*{Visão Geral da Leitura}
% Adicione aqui a visão geral da leitura

\section*{Capítulo 1}
No capítulo descrito, o autor usa um exemplo para ilustrar erros comuns em projetos de pesquisa. Um aluno de mestrado, tentando resolver um problema de mobilidade causado por um rio, falhou em seguir uma abordagem científica adequada. Aqui está um resumo dos erros cometidos:

1. \textbf{Falta de Revisão Bibliográfica}: O aluno não fez uma revisão adequada para encontrar soluções existentes para atravessar rios, focando em aspectos tangenciais como as propriedades da água e correntezas.

2. \textbf{Escolha de Métodos Inadequados}: Ao invés de explorar métodos comprovados, o aluno criou uma abordagem inovadora e arriscada—usando uma catapulta para lançar pessoas—sem justificar adequadamente essa escolha.

3. \textbf{Falta de Comparação com Soluções Existentes}: O aluno comparou os resultados obtidos apenas entre as suas próprias tentativas, sem considerar soluções já estudadas na literatura.

4. \textbf{Problema Local}: O aluno focou em um problema específico de um rio local sem avaliar se essa questão era relevante para outros contextos ou rios.

5. \textbf{Não Consultou o Orientador}: O aluno não procurou orientação adequada ao longo do processo, o que poderia ter ajudado a corrigir a abordagem e evitar erros.

Como resultado, apesar dos esforços e inovações, a pesquisa não foi bem-sucedida e o aluno foi reprovado. O exemplo sublinha a importância de seguir uma metodologia científica rigorosa, incluindo uma revisão bibliográfica abrangente, justificativa adequada para métodos escolhidos, comparação com soluções existentes e consulta com orientadores.

\section*{Capítulo 2}
Após introduzir a história que serviu de base para a reflexão do autor, o capítulo 2 aborda os diferentes estilos de pesquisa correntes na computação, elencando principalmente 5 estilos com diferentes aplicabilidades e objetivos, sendo os 4 primeiros estilos empíricos e o último necessitando de provas matemáticas/lógicas. Demonstra que a escolha do estilo deve ocorrer de acordo com a natureza do problema e o estado da arte da área de pesquisa.

\subsection*{Estilo "Apresentação de um produto"}
Tipo de pesquisa normalmente exploratória, focada na apresentação de algo novo em comparação com trabalhos anteriores. Abordagem vista como ingênua, utilizada mais em eventos focados em aplicações práticas e não tanto em publicações científicas relevantes.

\subsection*{Estilo "Apresentação de algo diferente"}
Este tipo é considerado mais amadurecido que o anterior, focado em apresentar uma nova forma de resolver o problema, comparando técnicas existentes, muitas vezes de maneira qualitativa. Trabalhos que utilizam estudos de caso e comparações superficiais podem ser aceitos em publicações, desde que os argumentos sejam convincentes, embora sem rigor científico comprovado. A hipótese de pesquisa é central, sendo necessário justificá-la e testá-la. O uso de tabelas comparativas ajuda a estruturar essas pesquisas, destacando a incorporação de características que diferenciam o novo artefato proposto das soluções anteriores.

\subsection*{Estilo "Apresentação de algo presumivelmente melhor"}
Em áreas um pouco mais maduras, novas abordagens devem ser comparadas quantitativamente com métodos da literatura, e a criação de benchmarks é muitas vezes necessária. No entanto, esse tipo de pesquisa exige cuidado, pois o pesquisador pode incorrer em erros ao testar tanto sua abordagem quanto as anteriores. Para garantir credibilidade, a comparação deve ser feita com métodos atuais, preferencialmente de até dois anos atrás, e a aplicação precisa ser detalhada, isolando fatores que possam influenciar os resultados. Além disso, é fundamental definir métricas claras e objetivas para validar as afirmações, tornando as comparações mais precisas e confiáveis.

\subsection*{Estilo "Apresentação de algo reconhecidamente melhor"}
O nível mais avançado da pesquisa científica envolve a apresentação de resultados empíricos baseados em testes padronizados e internacionalmente aceitos, o que garante maior credibilidade e reprodutibilidade. Nessa abordagem, o autor não precisa testar outras metodologias, mas sim comparar sua solução com dados e métricas já estabelecidos pela comunidade. Esse tipo de pesquisa, típico de boas teses de doutorado, visa avançar o estado da arte, tornando a nova solução uma referência no campo. Embora os testes e dados estejam disponíveis, o maior desafio é encontrar uma hipótese inovadora e promissora que justifique o estudo. Um exemplo em computação seria uma pesquisa que propõe um novo algoritmo de ordenação e o compara com os algoritmos mais eficientes da literatura, mostrando que ele é superior em termos de tempo e espaço.

\subsection*{Estilo "Apresentação de uma prova"}
Pesquisas que envolvem provas matemáticas são essenciais em subáreas da computação, como métodos formais e compiladores, onde é necessário apresentar demonstrações rigorosas de correção e eficiência. Nesses casos, a construção teórica deve ser clara, definindo conceitos e provando que sua aplicação leva a resultados específicos. Esses resultados podem incluir a prova da optimalidade de um algoritmo, a impossibilidade de resolver certos problemas com algoritmos ou a demonstração de limites de complexidade mínima para a solução de problemas específicos. Este tipo, apesar de ser o mais difícil de realizar, é o mais difícil de refutar.

\section*{Capítulo 3}
A preparação de um trabalho de pesquisa deve preceder a escrita propriamente dita. Muitos alunos, ansiosos por começar, cometem o erro de escrever sem ter realizado uma pesquisa substancial. É essencial distinguir a revisão bibliográfica, que compila conhecimentos existentes, da pesquisa científica, que gera novos conhecimentos. A revisão não deve ser extensa ou desconectada do objetivo principal da pesquisa, pois pode se tornar irrelevante e confusa. O sucesso de um trabalho depende de um objetivo bem definido, que orienta todo o processo de justificativa, metodologia, resultados e a própria revisão bibliográfica.

\subsection*{Seção 3.1}
Escolher um objetivo de pesquisa é um desafio crucial em mestrado ou doutorado. O objetivo deve ser claramente definido e normalmente envolve validar uma hipótese específica.
Normalmente, a descrição de um problema tem três etapas, que envolvem enunciá-lo precisamente, explicar que a questão ainda não foi tratada e justificar a importância da solução.
Assim, para se escolher um objetivo de pesquisa, é recomendado que se sigam 3 passos: escolher a área de conhecimento, revisar a bibliografia, definir o objetivo. Porém, este processo é iterativo, pois durante a pesquisa é possível alterar o objetivo ao saber mais sobre o problema e o tema.
Já o tema é normalmente definido ao restringir as áreas de onde você está pesquisando, como restringir sua pesquisa sobre métodos de busca a apenas o algoritmo A*. Outra forma de pensar o tema é, além dessa restrição, pensar em um problema específico, como a busca de caminhos em um labirinto, lembrando que a contribuição para a ciência no final deve ser à Computação, e não à área que contém o problema.
Quanto ao problema, a monografia deve apresentar uma solução para algo, aqui o autor não aprofunda além de utilizar o exemplo do primeiro capítulo.

\subsection*{Seção 3.2}

Esta seção versa sobre a importância da revisão bibliográfica, que tem como objetivo suprir as deficiências de conhecimento do pesquisador.
Normalmente é recomendado iniciar por trabalhos mais abrangentes, como livros (que normalmente contêm informações completas e amadurecidas) e revisões, e depois ir para artigos mais específicos (que são uma boa fonte de ideias de pesquisas) e considerados clássicos, que normalmente se encontram destacados nas revisões, e, por fim, para artigos mais recentes (que mostram o estado da arte). Uma boa revisão bibliográfica evita que a monografia reinvente a roda e dá fundamentação teórica para a pesquisa.
O autor recomenda a utilização de fichas de leitura como forma de organizar as informações obtidas, facilitando a escrita da monografia.
Algo específico apontado pelo texto sobre a área da computação é que artigos são bem vistos pela comunidade, visto que é uma área em constante evolução, e o tempo de revisão de uma publicação em periódico é muito longo, o que faz com que artigos sejam mais atualizados, mesmo que possam variar em termos de qualidade.
Durante a leitura, o texto recomenda que se exercite a crítica, questionando a validade dos argumentos, a qualidade dos resultados e a relevância do trabalho para a área de pesquisa. Este exercício normalmente leva a possíveis melhorias na pesquisa, gerando novas ideias. Outra recomendação é que o pesquisador deve constantemente se expor a resumos de novos artigos em conferências e periódicos renomados, para se manter atualizado.

\subsection*{Seção 3.3}

O objetivo da pesquisa é demonstrar que alguma hipótese é ou não verdadeira, como "demonstrar", "provar" ou "melhorar de acordo com métrica X", verbos estes que normalmente estão atrelados ao problema que se deseja resolver. Esse problema não deve ser demasiadamente trivial, nem demasiadamente complexo, e deve ser possível de ser resolvido com os recursos disponíveis, principalmente o tempo.
Para trabalhos de mestrado e doutorado, é obrigatório que a pesquisa contribua para o estado da arte da área de conhecimento, e não apenas a criação de um sistema com os conhecimentos adquiridos no curso. Esta abordagem é mais comum em trabalhos de graduação e especialização.
O objetivo também deve ser quebrado em sub-objetivos, que são mais específicos e fáceis de validar. Assim, o pesquisador vai conseguir, ao final da pesquisa, avaliar seu progresso e, ao dividir o objetivo, encontrar áreas que precisam de mais pesquisa ou aprimoramento.

\subsection*{Seção 3.4}

O método científico não deve ser confundido com a metodologia, que seria o estudo dos métodos. Em uma pesquisa, é comum que se escolha um método de trabalho e o utilize até comprovar ou refutar a hipótese, juntando evidências observáveis, empíricas, mensuráveis e analisando-as com o uso da lógica e da estatística. Logo, o método consistirá em uma sequência de passos para demonstrar que o objetivo foi atingido.

\subsubsection*{Subseção 3.4.1}

O método de trabalho pode conter a coleta e plotagem de dados, mas estes nunca devem ser o objetivo final da pesquisa, muito menos devem ser utilizados até que se saiba exatamente qual informação se está buscando.

\subsubsection*{Subseção 3.4.2}

O método de pesquisa pode ser dividido em duas grandes áreas. A experimental, onde o pesquisador manipula o ambiente estudado para observar como o ambiente reage, e a não experimental, onde o pesquisador apenas observa, tentando minimizar a interferência no ambiente e gerando conclusões a partir dessas observações.

\subsubsection*{Subseção 3.4.3}

Outra característica do método científico é a objetividade, que é a capacidade de se obter o mesmo resultado em diferentes experimentos, garantindo que observadores diferentes, assumindo um nível mínimo de competência, cheguem aos mesmos resultados. Além disso, é importante observar a qualidade das definições, que devem definir de maneira objetiva o fenômeno e convencer os demais de que ela é razoável.

\subsubsection*{Subseção 3.4.4}

O empirismo é a base do método científico e representa a capacidade de se obter conhecimento a partir da observação e da evidência coletada de forma sistemática e controlada. Para a ciência, o empirismo é a maneira sensata de observar o mundo, de forma que todo conhecimento deve ser verificável objetivamente se o fenômeno é real e se a conclusão é válida. Esta abordagem deve se basear em dados e medidas bem definidas, e o questionamento das conclusões é bem-vindo, pois a ciência é um processo de constante revisão.

\section*{Conclusões}

O livro apresenta uma visão abrangente e detalhada sobre os métodos de pesquisa em ciência da computação, destacando-se pela clareza na explicação dos conceitos e pela organização lógica dos capítulos. No entanto, observa-se que o autor recorre frequentemente a anedotas e analogias, o que, embora torne a leitura mais acessível, pode comprometer a precisão de algumas definições. Além disso, seria benéfico se o livro trouxesse mais exemplos de pesquisas realizadas especificamente na área de computação, além dos mencionados no primeiro capítulo. Isso enriqueceria ainda mais o conteúdo e proporcionaria uma visão mais completa das aplicações práticas dos métodos discutidos.

\end{document}
